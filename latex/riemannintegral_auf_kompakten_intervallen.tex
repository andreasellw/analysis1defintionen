\section{Riemannintegral auf kompakten Intervallen (139-162)}
 
Satz: Stetige Funktionen auf kompakten Intervallen sind integrierbar. Linearität des Riemannintegrals. Hauptsatz der Differential- und Integralrechnung. Definition der Stammfunktion. Partielle Integration. Substitutionsformel. Aufgaben vom Typ wie in den Beispielen 10.21, 10.23 und 10.25 lösen können.


\subsection{Linearität des Riemannintegrals (144)}
$\text{Let } a ,b \in \mathbb { R } ,a \leq b ,I : = [ a ,b ]$.

The integral is linear: More precicely if $f ,g \in \mathcal { R } ( I ,\mathbb { K } ) \text{ and } \lambda ,\mu \in \mathbb { K } ,\text{ then } \lambda f + \mu g \in \mathcal { R } ( I ,\mathbb { K } )$ and 
\begin{equation}
\int ( \lambda f + \mu g ) = \lambda \int _ { I } f + \mu \int _ { I } g
\end{equation}

%\begin{figure}[H]
%	\centering
%  \includegraphics[width=0.7\textwidth]{../img/{144-theorem-10-11}.png}
%	\caption{144-theorem-10-11}
%	\label{144-theorem-10-11}
%\end{figure}

\subsection{Hauptsatz der Differential- und Integralrechnung (152)}

\begin{leftbar}
$\text{ If } a ,b \in \mathbb { R } ,a \leq b ,I : = [ a ,b ] ,f : I \rightarrow \mathbb { C }$, then denote

\noindent\begin{minipage}{.5\linewidth}
\begin{equation}
\begin{split}
\int _ { a } ^ { b } f &: = \int _ { I } f, \\ 
[ f ( t ) ] _ { a } ^ { b } &: = [ f ] _ { a } ^ { b } : = f ( b ) - f ( a )
\end{split}
\end{equation}
\end{minipage}%
\begin{minipage}{.5\linewidth}
\begin{equation}
\begin{split}
\int _ { b } ^ { a } f &: = - \int _ { a } ^ { b } f, \\
[ f ( t ) ] _ { b } ^ { a } &: = [ f ] _ { b } ^ { a } : = f ( a ) - f ( b )
\end{split}
\end{equation}
\end{minipage}%

where $f \in \mathcal { R } ( I ,\mathbb { C } )$.

\end{leftbar}
%\begin{figure}[H]
%	\centering
%  \includegraphics[width=0.7\textwidth]{../img/{152-notation-10-18}.png}
%	\caption{152-notation-10-18}
%	\label{152-notation-10-18}
%\end{figure}

$\text{Let } a ,b \in \mathbb { R } ,a < b ,I : = [ a ,b ]$.
\begin{enumerate}
\item $\text{ If } f \in \mathcal { R } ( I ,\mathbb { K } )$ is continuous in $\xi \in I$, then, for each $c \in I$, the function
\begin{equation}
F _ { c } : I \rightarrow \mathbb { K } ,\quad F _ { c } ( x ) : = \int _ { c } ^ { x } f ( t ) d t,
\end{equation}
is \textbf{differentiable} in $\xi$ with $F _ { c } ^ { \prime } ( \xi ) = f ( \xi )$. In particular, if $f \in C ( I ,\mathbb { K } )$, then $F _ { c } \in C ^ { 1} ( I ,\mathbb { K } )$ and $F _ { c } ^ { \prime } ( x ) = f ( x )$ for each $x \in I$.

\item If $F \in C ^ { 1} ( I ,\mathbb { K } )$ or, alternatively, $F : I \rightarrow \mathbb { K }$ is differentiable with integrable derivative $F ^ { \prime } \in \mathcal { R } ( I ,\mathbb { K } )$, then 
\begin{equation}
F ( b ) - F ( a ) = [ F ( t ) ] _ { a } ^ { b } = \int _ { a } ^ { b } F ^ { \prime } ( t ) \text{d} t,
\end{equation}
and
\begin{equation}
F ( x ) = F ( c ) + \int _ { c } ^ { x } F ^ { \prime } ( t ) d t \quad \text{ for each } c ,x \in I.
\end{equation}

\end{enumerate}


%\begin{figure}[H]
%	\centering
%  \includegraphics[width=0.7\textwidth]{../img/{152-theorem-10-19}.png}
%	\caption{152-theorem-10-19}
%	\label{152-theorem-10-19}
%\end{figure}

\subsection{Definition der Stammfunktion (153)}

$\text{ If } I \subseteq \mathbb { R } ,f : I \rightarrow \mathbb { K } ,\text{ and } F : I \rightarrow \mathbb { R }$ is a differentiable function with $F ^ { \prime } = f$, then $F$ is called a \textit{primitive} or \textit{antiderivative }of $f$.

%\begin{figure}[H]
%	\centering
%  \includegraphics[width=0.7\textwidth]{../img/{153-definition-10-20}.png}
%	\caption{153-definition-10-20}
%	\label{153-definition-10-20}
%\end{figure}

\subsection{Partielle Integration (154)}

Let $a ,b \in \mathbb { R } ,a < b ,I : = [ a ,b ]$. If $f ,g \in C ^ { 1} ( I ,\mathbb { C } )$, then the following integration by parts formula holds:
\begin{equation}
\int _ { a } ^ { b } f g ^ { \prime } = [ f g ] _ { a } ^ { b } - \int _ { a } ^ { b } f ^ { \prime } g
\end{equation}

%\begin{figure}[H]
%	\centering
%  \includegraphics[width=0.7\textwidth]{../img/{154-theorem-10-22}.png}
%	\caption{154-theorem-10-22}
%	\label{154-theorem-10-22}
%\end{figure}

\subsection{Substitutionsformel (154)}
Let $I ,J \subseteq \mathbb { R }$ be intervals, $\phi \in C ^ { 1} ( I ) \text{ and } f \in C ( J ,\mathbb { C } )$. If $\phi ( I ) \subseteq J$, then the following change of variables formula holds for each $a ,b \in I$:
\begin{equation}
\int _ { \phi ( a ) } ^ { \phi ( b ) } f = \int _ { \phi ( a ) } ^ { \phi ( b ) } f ( x ) d x = \int _ { a } ^ { b } f ( \phi ( t ) ) \phi ^ { \prime } ( t ) d t = \int _ { a } ^ { b } ( f \circ \phi ) \phi ^ { \prime }
\end{equation}

%\begin{figure}[H]
%	\centering
%  \includegraphics[width=0.7\textwidth]{../img/{154-theorem-10-24}.png}
%	\caption{154-theorem-10-24}
%	\label{154-theorem-10-24}
%\end{figure}

\subsection{Aufgaben vom Typ wie in den Beispielen 10.21, 10.23 und 10.25}

\begin{itemize}
\item \textbf{Example 10.21} Due to the fundamental theorem, if we know a function's antiderivative,
we can easily compute its integral over a given interval. Here are three simple
examples:
\begin{equation}
 \int _ { 0} ^ { 1} \left( x ^ { 5} - 3x \right) \text{d} x = \left[ \frac { x ^ { 6} } { 6} - \frac { 3x ^ { 2} } { 2} \right] _ { 0} ^ { 1} = \frac { 1} { 6} - \frac { 3} { 2} = - \frac { 4} { 3}
 \end{equation} 
 \begin{equation}
\int _ { 1} ^ { e } \frac { 1} { x } \text{d} x = [ \ln x ] _ { 1} ^ { e } = \ln e - \ln 1= 1
 \end{equation} 
 \begin{equation}
\int _ { 0} ^ { \pi } \sin x \text{d} x = [ - \cos x ] _ { 0} ^ { \pi } = 2
 \end{equation} 

\item \textbf{Example 10.23} We compute the integral $\int _ { 0} ^ { 2\pi } \sin ^ { 2} t \text{d}t$:
\begin{equation}
\int _ { 0} ^ { 2\pi } \sin ^ { 2} t \text{d} t = [ - \sin t \cos t ] _ { 0} ^ { 2\pi } + \int _ { 0} ^ { 2\pi } \cos ^ { 2} t \text{d} t = \int _ { 0} ^ { 2\pi } \cos ^ { 2} t \text{d} t
\end{equation}
Adding$\int _ { 0} ^ { 2\pi } \sin ^ { 2} t \text{d} t$ on both sides and using $\sin ^ { 2} + \cos ^ { 2} \equiv 1$ yields
\begin{equation}
2\int _ { 0} ^ { 2\pi } \sin ^ { 2} t \text{d} t = \int _ { 0} ^ { 2\pi } 1 \text{d} t = 2\pi
\end{equation}
i.e. $\int _ { 0} ^ { 2\pi } \sin ^ { 2} t d t = \pi$.
 
 
\item \textbf{Example 10.25}  
We compute the integral $\int _ { 0} ^ { 1} t ^ { 2} \sqrt { 1- t } \text{d} t$ using the change of variables $x : = \phi ( t ) : = 1- t ,\phi ^ { \prime } ( t ) = - 1$:
\begin{equation}
\begin{split}
\int _ { 0} ^ { 1} t ^ { 2} \sqrt { 1- t } \text{d} t &= - \int _ { 1} ^ { 0} ( 1- x ) ^ { 2} \sqrt { x } \text{d} x = \int _ { 0} ^ { 1} \left( \sqrt { x } - 2x \sqrt { x } + x ^ { 2} \sqrt { x } \right) \text{d} x \\
 &= \left[ \frac { 2x ^ { \frac { 3} { 2} } } { 3} - \frac { 4x ^ { \frac { 5} { 2} } } { 5} + \frac { 2x ^ { \frac { 7} { 2} } } { 7} \right] _ { 0} ^ { 1} = \frac { 16} { 105}
\end{split}
\end{equation}

\end{itemize}

%\begin{figure}[H]
%	\centering
%  \includegraphics[width=0.7\textwidth]{../img/{153-example-10-21}.png}
%	\caption{153-example-10-21}
%	\label{153-example-10-21}
%\end{figure}

%\begin{figure}[H]
%	\centering
%  \includegraphics[width=0.7\textwidth]{../img/{154-example-10-23}.png}
%	\caption{154-example-10-23}
%	\label{154-example-10-23}
%\end{figure}

%\begin{figure}[H]
%	\centering
%  \includegraphics[width=0.7\textwidth]{../img/{155-example-10-25}.png}
%	\caption{155-example-10-25}
%	\label{155-example-10-25}
%\end{figure}