\section{Stetigkeit plus Zubehör}
 
Definition der Stetigkeit von Funktionen, die auf Teilmengen der komplexen (oder reellen) Zahlen definiert sind und in die reellen oder komplexen Zahlen abbilden. Folgenkriterium für Stetigkeit. Sie sollten einfache Stetigkeitsbeweise sowohl mit dem Epsilon-Delta-Kriterium als auch mit dem Folgenkriterium durchföhren können. Satz: Sind zwei Funktionen stetig, so auch Vielfache, die Summe, das Produkt, der Quotient, falls der Nenner nicht Null ist, der Betrag der Funktion sowie der Realteil und der Imaginärteil der Funktion. Satz: Eine komplexwertige Funktion ist genau dann stetig, wenn ihr Realteil und ihr Imaginärteil beide stetig sind. Satz: Betragsfunktion, Polynome und rationale Funktionen sind stetig, sofern der Nenner nicht Null ist. Satz: Die Komposition stetiger Funktionen ist stetig. Definition von beschränkten, abgeschlossenen und kompakten Teilmengen der komplexen Zahlen. Beschränkte Intervalle sind beschränkt, abgeschlossene Intervalle sind abgeschlossen. Offene und halboffene Intervalle sind nicht abgeschlossen. Nur Intervalle der Form [a,b] sind kompakt. Epsilon-Umgebungen sind beschränkt, aber nicht abgeschlossen. Endliche Vereinigungen und beliebige Durchschnitte erhalten Beschränktheit, Abgeschlossenheit und Kompaktheit. Endliche Mengen sind kompakt. Urbilder von abgeschlossenen Mengen unter stetigen Abbildungen sind abgeschlossen. Abgeschlossene Kreisscheiben und Kreise sind kompakt. Halbräume in den komplexen Zahlen sind abgeschlossen. Satz: Stetige Bilder kompakter Mengen sind kompakt. Definition globaler und lokaler Extrema ((strenge) Minima und Maxima). Satz: Stetige Abbildungen auf kompakten Mengen nehmen ihr (globales) Maximum und Minimum an. Nullstellensatz von Bolzano, Zwischenwertsatz, stetige Funktionen bilden Intervalle auf Intervalle ab. Definition der n.ten Wurzel einer nichtnegativen Zahl; die zugehörige Funktion ist stetig und streng monoton steigend. Nicht rationale Zahlen heißen irrational; die Menge der rationalen Zahlen ist abzählbar; die Menge der irrationalen Zahlen ist nicht abzählbar. Definition der Dichtheit einer Menge in den reellen Zahlen. Satz: Die rationalen Zahlen sind dicht in den reellen Zahlen; die irrationalen Zahlen sind ebenfalls dicht. Satz: Jede reelle Zahl ist der Grenzwert einer streng steigenden Folge rationaler Zahlen und einer streng fallenden Folge rationaler Zahlen. Definition von Potenzen mit nichtnegativer Basis und reellen Exponenten; es gelten die üblichen Potenzgesetze. Definition von allgemeinen Potenzfunktionen und Exponentialfunktionen. Satz: Potenzfunktionen sind auf ihrem jeweiligen Definitionsbereich stetig, sowie streng steigend für positiven und streng fallend für negativen Exponenten. Satz: Exponentialfunktionen sind stetig sowie streng steigend für Basis a>1 und streng fallend für Basis 0 < a < 1. Definition des Logarithmus, speziell des natürlichen Logarithmus, Logarithmengesetze gemäß Th. 7.75. 